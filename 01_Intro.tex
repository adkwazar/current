
\title[BT2-212: Wprowadzenie]{Bioinformatyka 2 - kurs mały}
\subtitle{Wprowadzenie i sprawy organizacyjne.\newline
          Podstawy bioinformatyki sekwencji.}
\author{Krzysztof Murzyn}
\institute{%
Zakład Biofizyki Obliczeniowej i Bioinformatyki\\ Wydział Biochemii,
Biofizyki i Biotechnologii\\ Uniwersytet Jagielloński}
\date{}

\begin{document}

\begin{frame}
 \titlepage
\end{frame}

\only<presentation>{\begin{frame}
 \frametitle{Plan wykładu}
 \tableofcontents
\end{frame}}

\maketitle

\begin{frame}
\frametitle{Kontakt}\scriptsize
dr hab. Krzysztof Murzyn\\
Zakład Biofizyki Obliczeniowej i Bioinformatyki\\
WBBiB UJ, 
pok. B028 (\wgre{wtorek 11:00-11:45}), \wor{Teams} rozmowa / czat\\
tel. (12) 664-63-79\\
\wgre{email: krzysztof.murzyn@uj.edu.pl}\\
{\wbl{http://bioinfo.mol.uj.edu.pl/modmol/People/KrzysztofMurzyn}}

\end{frame}

\section{Podstawowe informacje}

\subsection{Terminy zajęć}

\begin{frame}
\frametitle{Zajęcia i warunki zaliczenia}

\scriptsize

 \begin{itemize} \item 5 wykładów (łącznie 10h), czwartek 10:30-12,
 sala 1.01.13, \wor{stacjonarny} test
 pojedynczego wyboru ($+2/-1/0$) z pytaniami otwartymi i zamkniętymi

 \item 5 ćwiczeń (po 3h, 135 min), \wor{stacjonarny} test praktyczny
       (zestaw zadań do samodzielnego rozwiązania) - łącznie 20h,
       wszystkie ćwiczenia stacjonarnie

 \item ćwiczenia rozpoczynają się w tygodniu po pierwszym wykładzie,
 koordynatorem ćwiczeń (harmonogram, oceny, etc) jest mgr Adrian Kania
\end{itemize}

\begin{minipage}[t]{.65\textwidth}\vspace{0pt}
\begin{itemize}
 \item zaliczenie kursu
 obejmuje: \begin{itemize}\scriptsize \item \wbl{zaliczenie ćwiczeń}\\
 wykonanie ćwiczeń ($5 \cdot 6$~pkt), wynik testu praktycznego
 (120 min, ok. 5 zadań, $70$~pkt); łącznie: $100$~pkt (ZAL
 (50+)/NZAL) \item \wbl{zaliczenie wykładu} -- ocena do średniej,\\
 warunek wstępny: zaliczenie ćwiczeń, wynik punktowy zaliczenia
 ćwiczeń (maks. $100$ pkt) powiększony o wynik testu pojedynczego
 wyboru zawierającego pytania z zagadnień poruszanych na wykładach i
 ćwiczeniach (teoria, maks. $100$ pkt); łącznie:
 $200$~pkt \end{itemize} \end{itemize}
\end{minipage}\hfill
\begin{minipage}[t]{.34\textwidth}\vspace{0pt}
\scriptsize\centering\renewcommand{\arraystretch}{1.2}
 \begin{tabular} {r | r}
\multicolumn{2}{c}{Procentowa skala ocen}\\
\hline
bdb & $\ge$ 90 \%\\
+db & $[\; 80, 90 \;)$ \%\\
db &  $[\; 70, 80 \;)$ \%\\
+dst & $[\; 60, 70 \;)$ \%\\
dst & $[\; 50, 60 \;)$ \%\\
ndst & $< 50$ \%
 \end{tabular}
\end{minipage}
\end{frame}

%% \begin{frame}
%% \frametitle{Przebieg ćwiczeń}
%% \scriptsize
%% \begin{itemize}

%%  \item niezależnie od tego czy zajęcia przebiegają zdalnie czy
%%  stacjonarnie, studenci pracują w tzw. \wbl{zespołach ćwiczeniowych}
%%  złożonych z 3-4 os. (zależnie od wielkości grupy zajęciowej (max. 16
%%  os))

%%  \item \wgre{\bf ćwiczenie 1}: wprowadzenie, organizacja środowiska
%%  pracy (WSL/BASH, edytor tekstu (Win10: SublimeText4 lub VSCode,
%%  MacOS: CotEditor), praca zdalna ssh/scp/sftp, polecenia powłoki
%%  tekstowej), rozwiązywanie przykładowych zadań, \wpi{\bf prezentacja
%%  ćwiczeniowca} dot. zagadnień do opracowania na kolejne
%%  ćwiczenia, \wpi{\bf zadanie domowe}: omówienie zestawu ok. 8
%%  pytań / zadań do przygotowania przez każdego studenta w
%%  ramach \wbl{zespołów ćwiczeniowych}

%%  \item \wgre{\bf ćwiczenia 2 --
%%  5}: \begin{description} \item[\wpi{rozwiązywanie zadań}] (ok. 70$-$80
%%  min) ćwiczeniowiec wybiera studenta z \wbl{zespołu ćwiczeniowego} do
%%  rozwiązania wskazanego zadania w trybie \emph{na żywo} (wskazana
%%  osoba omawia rozwiązanie, pozostałe osoby z zespołu są aktywne na
%%  czacie); \wpi{oceniny jest zespół jako {\bf całość}} ; w trakcie
%%  ćwiczeń każdy \wbl{zespół ćwiczeniowy} ma możliwość przedstawienia
%%  rozwiązania dokładnie dwóch zadań \item[\wpi{prezentacja
%%  ćwiczeniowca}] nowe zagadnienia do przygotowania na kolejne
%%  ćwiczenia \item[\wpi{zadanie domowe}] treści zadań do rozwiązania
%%  na kolejne ćwiczenia \end{description}

%%  \item \wgre{\bf ćwiczenie 6}: \wpi{\bf rozwiązywanie zadań domowych}
%%  oraz krótki \wpi{\bf test ćwiczeniowy} (20$-$30~min, 2
%%  zadania, \wbl{praca indywidualna}), omówienie przygotowań do test
%%  praktycznego

%% \end{itemize}
%% \end{frame}

%% \begin{frame}
%% \frametitle{Zaliczanie zadań problemowych}

%% \begin{itemize}\scriptsize

%%  \item wybrany przedstawiciel \wbl{grupy ćwiczeniowej} udostępnia
%%  pulpit i omawia poszczególne etapy rozwiązywania konkretnego zadania,
%%  interpretując na bieżąco uzyskiwane wyniki (zatem \wor{\bf należy
%%  unikać prezentowania wyników uzyskanych przed ćwiczeniami})

%%  \item pozostali członkowie \wbl{grupy ćwiczeniowej} mogą udzielać
%%  wskazówek raportującemu studentowi lub uzupełniać jej / jego
%%  wypowiedź, ale wyłącznie z wykorzystaniem \wor{\bf czata Teams}
%%  (wiadomości widoczne dla wszystkich uczestników ćwiczeń)

%%  \item student raportujący rozwiązanie zadania nie może przekroczyć
%%  czasu wskazanego przez ćwiczeniowca (zazwyczaj 6 do 8 min) --
%%  ograniczony czas zapewnia zachowanie odpowiedniego stopnia trudności
%%  rozwiązywanego zadania i wymaga od studenta dobrego przygotowania

%%  \item ćwiczeniowiec ocenia przedstawione rozwiązanie
%%  zadania; \wor{\bf każdy członek zespołu otrzymuję taką samą ocenę}; w
%%  ten sposób, ocena za rozwiązywane zadanie odzwierciedla
%%  m.in. rozwijanie umiejętności pracy zespołowej a cały system
%%  przygotowania do zajęć zachęca do dzielenia się wiedzą i
%%  umiejętnościami

%% \end{itemize}

%% \centering\rule{.8\textwidth}{.5pt}\scriptsize

%%  oceny cząstkowe z kursu (ćwiczenia, test próbny, test praktyczny,
%%  test pojedynczego wyboru, końcowa ocena z ćwiczeń i wykładu dostępne
%%  w systemie USOSweb (moduł Sprawdziany)

%% \end{frame}

\subsection{Zawartość kursu}

\begin{frame}\frametitle{Wykłady}
\begin{enumerate}\scriptsize
 
 \item Wprowadzenie i sprawy organizacyjne. Podstawy bioinformatyki
 sekwencji. Wprowadzenie do programowania w Pythonie. Pozyskiwanie i
 przetwarzanie danych biologicznych. [Murzyn/Kania]

 \item Metody przewidywania i walidacji struktury przestrzennej
 białek. Metaserwery predykcyjne. [Murzyn]
 
 \item Zaawansowane protokoły obliczeniowe w modelowaniu molekularnym
 biocząsteczek. [Murzyn]

 \item Techniki nauczania maszynowego w zastosowaniu do analizy danych
 z mikromacierzy. [Kania]

 \item Techniki przetwarzania i analizy danych z sekwencjonowania
 nowej generacji. Potoki analityczne na serwerze Galaxy. [Majta]

\end{enumerate}

\centering\rule{.8\textwidth}{.5pt}\scriptsize

Prezentacje z wykładów (PDF) będą na bieżąco udostępniane
w~materiałach kursowych w~dedykowanym zespole
Teams.

\end{frame}


\begin{frame}\frametitle{Ćwiczenia}
\begin{enumerate}\scriptsize
 
 \item Pozyskiwanie i przetwarzanie danych biologicznych z
 wykorzystaniem technik programowania w Pythonie. [Kania]

 \item Metody przewidywania i walidacji struktury przestrzennej
 białek. Metaserwer predykcyjne. [Gucwa]

 \item Zaawansowane protokoły obliczeniowe w modelowaniu molekularnym
 biocząsteczek. [Gucwa]

 \item Techniki nauczania maszynowego w zastosowaniu do analizy danych
 z mikromacierzy. [Kania]

 \item Techniki przetwarzania i analizy danych z sekwencjonowania
 nowej generacji. Serwer Galaxy. [Majta]

\end{enumerate}

\centering\rule{.8\textwidth}{.5pt}\scriptsize

Materiały do ćwiczeń udostępniane przez ćwiczeniowca.

\end{frame}


%% \subsection{Biblioteka użytkownika kursu}

%% \begin{frame}\frametitle{Interesujące książki}

%% \begin{center}\begin{tabular}{c c c c}
%% \includegraphics[width=.2\textwidth]{images/book_baxevanis_pwn.jpg} & 
%% \includegraphics[width=.2\textwidth]{images/book_attwood-higgs_pwn.jpg} &
%% \includegraphics[width=.2\textwidth]{images/book_barry.jpg} & 
%% \includegraphics[width=.2\textwidth]{images/book_xiong.jpg}
%% \end{tabular}\end{center}

%% \raggedright
%% \begin{itemize}\scriptsize

%%  \item \wbl{Bioinformatics - A Practical Guide to the Analysis of
%%  Genes and Proteins}, AD Baxevanis \& BFF Outellette, \wgo{*}

%%  \item \wbl{Bioinformatics and Molecular Evolution}, T Attwood, P
%%  Higgs, \wgo{****}

%%  \item \wbl{Łatwe drzewa filogenetyczne}, H. Barry, \wgo{**}

%%  \item \wbl{Podstawy bioinformatyki}, Xiong Jin, \wgo{**}


%% \end{itemize}
%% \end{frame}

\subsection{Cele kursu}

\begin{frame}
\frametitle{Wiedza i umiejętności}
\scriptsize

\wbl{\small Wiedza}

\begin{itemize}

 \item wybrane zagadnienia bioinformatyki sekwencji oraz
 bioinformatyki strukturalnej

 \item pozyskiwanie i przetwarzanie danych biologicznych

 \item przewidywanie i walidacja struktury przestrzennej białek
 (modelowanie porównawcze, metaserwer predykcyjny iTasser)

 \item wybrane protokoły obliczeniowe w modelowaniu molekularnym

 \item techniki nauczania maszynowego w zastosowaniach bioinformatycznych

 \item techniki przetwarzania i analizy danych z sekwencjonowania
 nowej generacji (serwer Galaxy)

\end{itemize}

\wbl{\small Umiejętności}

\begin{itemize}

 \item uruchamianie i modyfikowanie prostych programów w języku Python
 v3

 \item obsługa specjalistycznego oprogramowania
 bioinformatycznego: \wpi{Jupyter}, \wpi{Modeller}, \wpi{Gromacs}

 \item korzystanie ze specjalistycznych serwisów internetowych,
 m.in. \wpi{NCBI Entrez}, \wpi{iTasser}, \wpi{Galaxy}

\end{itemize}
\end{frame}

\subsection{Praktyczne wskazówki}

\begin{frame}
\frametitle{macOS}

\begin{minipage}[t]{.4\textwidth}\vspace{0pt}

 \scriptsize Terminal
 tekstowy\\[1ex] \includegraphics[width=\textwidth]{images/MacOS-search}
 \includegraphics[width=\textwidth]{images/MacOS-terminal-app}
 \includegraphics[width=\textwidth]{images/MacOS-terminal}

\end{minipage}\hfill%
\begin{minipage}[t]{.59\textwidth}\vspace{0pt}

\begin{itemize}\scriptsize

 \item  Edytor tekstu (np. CotEditor, coteditor.com)
 \includegraphics[width=\textwidth]{images/coteditor}]

\end{itemize}

\vspace*{-4ex}\begin{itemize}\scriptsize
 \item Zmiana domyślnego programu powłoki (zsh) na coś innego (np. bash,
 csh): \wbl{chsh -s /bin/bash}.
\end{itemize}
\end{minipage}
\end{frame}

%% \begin{frame}
%% \frametitle{macOS: Xcode and Homebrew}

%% \scriptsize\raggedright
%% Instalacja \wgre{Xcode command line tools}
%% oraz \wgre{Homebrew} \wgre{package management system} umożliwiają
%% wygodne instalowanie pakietów oprogramowania z linii poleceń
%% (np. \wbl{brew install emboss})
%% \begin{minipage}[t]{.49\textwidth}\vspace{0pt}
%% \scriptsize\raggedright
%% Xcode jest zintegrowanym środowiskiem programistycznym dla macOS\\[1ex]

%% \wgre{https://developer.apple.com/xcode/}

%% \centering
%% \includegraphics[width=.5\textwidth]{images/xcode}
%% \end{minipage}\hfill%
%% \begin{minipage}[t]{.49\textwidth}\vspace{0pt}
%% \centering
%% \includegraphics[width=.6\textwidth]{images/homebrew}
%% \end{minipage}
%% \end{frame}

\begin{frame}
\frametitle{Windows 10+}

\begin{itemize}\scriptsize

 \item Windows Subsystem for Linux,\newline
 instalacja: \wgre{https://docs.microsoft.com/en-us/windows/wsl/install},\newline
 dobre
 praktyki: \wgre{https://docs.microsoft.com/en-us/windows/wsl/setup/environment}

 \includegraphics[width=.6\textwidth]{images/microsoft-store}

 \item edytor tekstu (VSCode (dla chcących programować), SublimeText4
 (lekki i~minimalistyczny))

 %% \item Miniconda
 %% (\wgre{https://docs.conda.io/en/latest/miniconda.html}) (łatwa
 %% instalacja m.in. pymola), izolowane środowiska programistyczne,
 %% menager pakietów oprogramowania

 %% \item ClustalX (\wgre{http://www.clustal.org/clustal2/})


\end{itemize}
\end{frame}

\begin{frame}[fragile]
\frametitle{MacOSX/Win10+/Linux: CONDA}


\begin{itemize}\scriptsize

 \item Conda otwartoźródłowy systemem zarządzania
 pakietami oprogramowania dostępny na \wpi{conda.io} dla systemów
 Win10+, Linux i MacOSX napisany w języku Python

 \item Conda umożliwia tworzenie izolowanych środowisk
 programistycznych (dowolny język programowania) oraz
 instalowanie/aktualizowanie/usuwanie pakietów oprogramowania z
 uwzględnieniem zależności między nimi

 \item dystrybucja pakietów realizowana jest w kanałach (ang. channels), w
 których zgrupowano (zazw. tematycznie) różnorodne oprogramowanie
 
 \item minimalistyczną wersją systemu Conda jest Miniconda (przy
 instalacji wybiera się domyślną wersję interpretera Pythona,
 np. Python v3.11); wersje interpretera można późnie łatwo zmienić
 (upgrade/downgrade)

 \item wyszukiwanie pakietów oprogramowania można zrealizować z linii
 poleceń (np. \wor{conda search pymol}) lub on-line na
 stronie \wpi{anaconda.org} (Anaconda to Conda poszerzona o zestaw
 pakietów Pythona (głównie z zakresu Danetyki (Data Science),
 wzgl. Inżynierii i Analizy Danych)

 \item użytkownicy Windows: warto zainstalować \wpi{WindowsTerminal}
 (Microsoft Store, free)

\end{itemize}

\begin{Verbatim}[fontsize=\relsize{-3},frame=single,framesep=1mm]
$ conda --version
$ conda create --name wbbib-python   # no i np. niestety nie ta wersja pythona, którą chcemy
$ conda activate wbbib-python
$ python --version
$ conda search python                # sprawdźmy, jakie wersje pythona dostępne
$ conda install python=3.8.3         # w bieżącym środowisku instalujemy to co potrzebujemy
$ python --version
$ conda deactivate wbbib-python
\end{Verbatim}

\end{frame}



\section{Podstawy bioinformatyki sekwencji}

\begin{frame}
\frametitle{Podstawy bioinformatyki sekwencji}

\small
\begin{minipage}[t]{.4\textwidth}\vspace{0pt}
 \centering Porównywanie
 sekwencji

 \vspace*{2em}\includegraphics[width=.9\textwidth]{images/seq-align-matrix}
\end{minipage}%
\begin{minipage}[t]{.6\textwidth}\vspace{0pt}
\centering Algorytmy 
 heurystyczne

 \vspace*{2em}\includegraphics[width=.9\textwidth]{images/blast-web}
\end{minipage}

\end{frame}

\subsection{Porównywanie sekwencji}

\begin{frame}[fragile]
\frametitle{Zapis sekwencji: format FASTA}

%%\begin{Verbatim}[fontsize=\scriptsize,
%%                 frame=single,framesep=3mm,commandchars=\\\{\}]
%%\wgo{>}\wor{gi|730028}|\wgre{sp}|\wbl{P40692}|\wpi{MLH1\_HUMAN} \wgo{DNA MISMATCH REPAIR PROTEIN MLH1}
\begin{Verbatim}[fontsize=\tiny, frame=single, framesep=3mm, commandchars=\\\{\}]
> gi|730028 | sp | P40692 | MLH1_HUMAN DNA MISMATCH REPAIR PROTEIN MLH1 
MSFVAGVIRRLDETVVNRIAAGEVIQRPANAIKEMIENCLDAKSTSIQVIVKEGGLKLIQ
IQDNGTGIRKEDLDTTSKLQSFEDLASISTYGFRGEALASISHVAHVTITTKTADGKCAY
RASYSDGKLKAPPKPCAGNQGTQITVEDTRRKALKNPSEEYGKILEVVGRYSVHNAGISF
SVKKQGETVADVRTLPNASTVDNIRSIFGNAVSRELIEIGCEKMNGYISNANYSVKKCIF
LLFINHRLVESTSLRKAIETVYAAYLPKNTHPFLYLSLEISPQNVDVNVHPTKHEV
\end{Verbatim}

\begin{description}\scriptsize
 \item[\tb{\wor{gi|730028}}] numer identyfikacyjny sekwencji
 
 \item[\tb{\wgre{sp}}] kod bazy danych sekwencji (np. gb, emb, pdb)

 \item[\tb{\wbl{P40692}}] kod dostępowy, alfanumeryczny identyfikator
 rekordu bazy danych (zalecane używanie w cytowaniach)

 \item[\tb{\wpi{MLH1\_HUMAN}}] nazwa rekordu: alfanumeryczny kod
 (max. 10 znaków) identyfikujący sekwencję; nie uwzględniany w wielu
 formatach, stąd nie zaleca się stosować go jako identyfikatora

 \item[\tb{\wgo{DNA MISMATCH..}}] zwięzły opis sekwencji
\end{description}
\end{frame}

\begin{frame}[fragile]
\frametitle{Homologia, konwergencja czy przypadek}
\noindent\begin{minipage}[c]{.4\textwidth}
\raggedright\relsize{-3}
\begin{itemize}
 \item podobieństwo sekwencji może wynikać z (1) pełnienia przez nie
 podobnych funkcji w podobnym środowisku (białka błonowe), (2)
 wspólnego pochodzenia ewolucyjnego, lub (3) przypadku

 \item homologia: opis relacji między genami / białkami (sekwencjami)
 wskazujący na ich wspólne pochodzenie ewolucyjne (własność binarna,
 przechodność homologii)
 
 \item sekwencje \wgo{homologiczne} łączy zależność \wgre{ortologii}
 jeśli ich ostatni przodek istniał w momencie specjacji

 \item jeśli ostatni wspólny przodek istniał w chwili duplikacji genu w
 genomie przodka, wtedy obie sekwencje łączy zależność \wgre{paralogii}
 
\end{itemize}
\end{minipage}\vspace{.01\textwidth}%
\begin{minipage}[c]{.59\textwidth}
\includegraphics[width=\textwidth]{images/ortopara}
\end{minipage}
\hrule

\begin{center}
\relsize{-2}białko zaangażowane w poreplikacyjną korektę DNA

\vspace*{2em}\begin{minipage}{.8\textwidth}
\tiny\texttt{%
\wgre{MSH2\_HUMAN}~~~\wor{626}~EK-{}GQGRIILKAS\wgo{RH}ACV\wgo{E}VQDEIA\wgo{FI}P\wgo{N}DVYFEKDKQMFH\wgo{IITGPNMGGKSTY}I\wgo{RQ}TGV~\wor{684}\\
\wgre{MSH2\_YEAST}~~~\wor{644}~PMDSERRTHLISS\wgo{RH}PVL\wgo{E}MQDDIS\wgo{FI}S\wgo{N}DVTLESGKGDFL\wgo{IITGPNMGGKSTY}I\wgo{RQ}VGV~\wor{703}\\
\wgre{MUTS\_ECOLI}~~~\wor{574}~DK-{}-{}-{}PGIRITEG\wgo{RH}PVV\wgo{E}QVLNEP\wgo{FI}A\wgo{N}PLNLSPQRR-{}ML\wgo{IITGPNMGGKSTY}M\wgo{RQ}TAL~\wor{629}}\\
\verb|                          :  .**. :*   : .**.* : :.  :  : *************:**..:|
\end{minipage}
\end{center}
\end{frame}

\begin{frame}
\frametitle{Złudne \wpi{podobieństwo} sekwencji}

\begin{tabular}{l l}
\wor{HBA\_HUMAN} & łańcuch $\alpha$ globin, \wscgo{człowiek}\\
\wor{HBB\_HUMAN} & łańcuch $\beta$ globin, \wscgo{człowiek}\\
\wor{LGB2\_LUPLU} & leghemoglobina, \wscgo{łubin}\\
\wbl{GTA1\_CAVPO} & \wpi{S-transferaza glutationu}, \wscgo{świnka morska}\\
\end{tabular}\vfill

\relsize{-2}\ttfamily
\begin{tabular}{r l | r}
\wor{HBA\_HUMAN} &  HGSAQVKGHGKKVADALTNAVAHVDDMPNAL & \wbl{38.7\%}\\
           & ~|:~:||~|||||~~|~::~:||:|::~~~~ & \wpi{64.5\%}\\
\wor{HBB\_HUMAN} &  MGNPKVKAHGKKVLGAFSDGLAHLDNLKGTF &\\[1ex]

\wor{HBA\_HUMAN} &  HGSAQVKGHGKKVADALTNA-{}-{}-{}-{}-VAHVDDMPNAL & \wbl{22.5\%}\\
          & ~~:~:::~|~~||~~~:~~|~~~~~|~~|~~~~~~| & \wpi{38.7\%}\\
\wor{LGB2\_LUPLU} & GNNPELQAHAGKVFKLVYEAAIQLQVTGVVVTDATL & \\[1ex]

\wor{HBA\_HUMAN} &  HGSAQVKGHGKKVAD-ALTNAVAHVDDMPNAL & \wbl{29.0\%}\\
          &  ||~~~:~|:~~~~||~~||~~:~~|::~~~:| & \wpi{48.4\%}\\         
\wbl{GTA1\_CAVPO} & HGQGYLVGNKLSKADILLTELLYMVEEFDASL &\\[1ex]
\end{tabular}

\end{frame}

\begin{frame}
\frametitle{Powtórzenia domen w białku SLIT}

\begin{minipage}[t]{.38\textwidth}\vspace{0pt}
\raggedright\relsize{-2}
białko SLIT\newline (\emph{Drosophila melanogaster}):\\[1ex]

24\wgo{LRR} (\emph{Leucine-Rich-Repeats}),\newline
7~\wgo{EGF}\\[1ex]
\wgre{dotmatcher}\newline \texttt{W13:T20 blosum62}
\end{minipage}%
\begin{minipage}[t]{.6\textwidth}\vspace{0pt}
\vspace*{-3ex}\includegraphics[width=\textwidth]{images/dotplot-ex1}
\end{minipage}
\end{frame}

\begin{frame}
\frametitle{Identyfikacja LCR}

\begin{minipage}[t]{.5\textwidth}\vspace{0pt}
\includegraphics[width=\textwidth]{images/LCR}
\end{minipage}%
\begin{minipage}[t]{.48\textwidth}\vspace{0pt}
\raggedright\scriptsize
\begin{itemize}

 \item sekwencje o niskiej złożoności składu (LCR, ang.\ \emph{Low
 Complexity Regions}) to odcinki, które cechuje nietypowy skład
 aminokwasowy/nukleotydowy

 \item identyfikację odcinków LCR często prowadzi się w oparciu o
 tzw.\ parametr złożoności składu $K$, który można wyrazić na wiele
 sposobów, np.\ dla:

 {\relsize{1}\[ K = \frac{1}{L} \log_N \frac{L!}{\prod_i n_i!}
 \]}

 parametr $K$ będzie przyjmował wartości między 0.0 (niska złożoność
 składu) a 1.0 (wysoka złożoność składu)

\end{itemize}
\end{minipage}
\end{frame}

\begin{frame}
\frametitle{Dopasowanie lokalne}

\raggedright\scriptsize
\begin{tabular}{l l}
\\ \wscgo{Cel}: & \parbox[t]{.8\textwidth}{\raggedright%\
identyfikacja krótkich fragmentów sekwencji o możliwie największym
podobieństwie}\\
\\[1ex]
\wscgo{Zastosowania:} & \parbox[t]{.8\textwidth}{\raggedright%\
domeny białkowe\\
egzony w genomowym DNA}\\
\\[1ex]
\end{tabular}

\noindent\begin{center}\begin{tabular}{ c }
\wscgo{Realizacja:}\\
\\[-3ex]
%\setlength{\fboxsep}{5mm}
\fbox{\parbox[t]{.7\textwidth}{\raggedright\begin{itemize}
\item zmodyfikowany algorytm dopasowania częściowego
\item przy obliczaniu \wgre{$\mathcal{F}$}, za każdym razem, gdy
\wgre{$\mathcal{F}(i,\,j) < 0 \Rightarrow \mathcal{F}(i,\,j) = 0$}, co
oznacza rozpoczynanie nowego \wbl{lokalnego} dopasowania
\item w kompletnej macierzy \wgre{$\mathcal{F}$: $\forall_{i,\,j}\;
\mathcal{F}(i,\,j) \ge 0$}
\item rekonstrukcja dopasowania (\emph{backtracing}) zaczyna się od
największej wartości \wgre{$\mathcal{F}$} a kończy w pozycji, gdzie
\wgre{$\mathcal{F}(i,j) = 0$}
\end{itemize}}\rule{1em}{0pt}}\\
\end{tabular}\end{center}\vspace{\stretch{3}}
\end{frame}

\begin{frame}
\frametitle{Dopasowanie lokalne: algorytm}

\vspace*{-1em}\noindent\small
\wgo{\relsize{2}\[ \mathcal{F}(i,j) = \max \left\{ \begin{array}{l}
\mathcal{F}(i-1,\,j-1) \pm 1\\
\mathcal{F}(i-1,\,j) -2\\
\mathcal{F}(i,\,j-1) -2\\
0
\end{array} \right. \]}

\framesubtitle{\wgre{Programowanie dynamiczne i \emph{backtracking}}}
\begin{minipage}[c]{.6\textwidth}\vspace{0pt}
\ttfamily\bfseries\relsize{2}
\begin{tabular}{ l | *{10}{r@{\hspace{.5em}}}}
\multicolumn{2}{c}{} & T & G & A & T & A & G & G & A & C\\
\cline{2-11}
   &  \wor{0} &  \wor{0} &  \wor{0} &  \wbl{0} &  \wor{0} &  \wor{0} &  \wor{0} &  \wor{0} &  \wor{0} &  \wor{0}\\
T  &  \wor{0} &  1 &  0 &  0 &  \wbl{1} &  0 &  0 &  0 &  0&   0\\
A  &  \wor{0} &  0 &  0 &  1 &  0 &  \wbl{2} &  0 &  0 &  1&   0\\
G  &  \wor{0} &  0 &  1 &  0 &  0 &  0 &  \wbl{3} & 1 &  0 &   0\\
C  &  \wor{0} &  0 &  0 &  0 &  0 &  0 &  1 &  2 &  0&   1\\
\end{tabular}\end{minipage}\hspace{.03\textwidth}%
\begin{minipage}[t]{.3\textwidth}\vspace{0pt}
\ttfamily\bfseries\relsize{3}\color{greenish}
\begin{tabular}{*{10}{r@{\hspace{.2em}}}}
T & G & A & T & A & G & G & A & C\\
- & - & - & T & A & G & - & - & -
\end{tabular}
\end{minipage}\vspace{\stretch{3}}
\end{frame}

\begin{frame}
 \frametitle{Macierz PAM250}

\tiny
\vspace*{-6ex}\begin{minipage}[t]{.2\textwidth}\vspace{0pt}
\sffamily\renewcommand{\arraystretch}{1.2}
\begin{tabular}{| l | r | *{5}{r@{\hspace*{1em}}} | *{4}{r@{\hspace*{1em}}} | *{3}{r@{\hspace*{1em}}} | *{4}{r@{\hspace*{1em}}} | *{3}{r@{\hspace*{1em}}}}
\cline{1-1}
\wor{Cys} & \multicolumn{1}{r}{\wgre{12}}\\
\cline{1-2}
\wor{Ser} &  0 &  \wgre{2}\\
\wor{Thr} & -2 &  \wgre{1} &  \wgre{3}\\
\wor{Pro} & -3 &  \wgre{1} &  \wgre{0} &  \wgre{6}\\
\wor{Ala} & -2 &  \wgre{1} &  \wgre{1} &  \wgre{1} &  \wgre{2}\\
\wor{Gly} & -3 &  \wgre{1} &  \wgre{0} & \wgre{-1} &  \wgre{1} &  \multicolumn{1}{r}{\wgre{5}}\\
\cline{1-7}
\wor{Asn} & -4 &  1 &  0 & -1 &  0 &  0 &  \wgre{2}\\
\wor{Asp} & -5 &  0 &  0 & -1 &  0 &  1 &  \wgre{2} &  \wgre{4}\\
\wor{Glu} & -5 &  0 &  0 & -1 &  0 &  0 &  \wgre{1} &  \wgre{3} &  \wgre{4}\\
\wor{Gln} & -5 & -1 & -1 &  0 &  0 & -1 &  \wgre{1} &  \wgre{2} &  \wgre{2} &  \multicolumn{1}{r}{\wgre{4}}\\
\cline{1-11}
\wor{His} & -3 & -1 & -1 &  0 & -1 & -2 &  2 &  1 &  1 &  3 &  \wgre{6}\\
\wor{Arg} & -4 &  0 & -1 &  0 & -2 & -3 &  0 & -1 & -1 &  1 &  \wgre{2} &  \wgre{6}\\
\wor{Lys} & -5 &  0 &  0 & -1 & -1 & -2 &  1 &  0 &  0 &  1 &  \wgre{0} &  \wgre{3} & \multicolumn{1}{r}{\wgre{5}}\\
\cline{1-14}
\wor{Met} & -5 & -2 & -1 & -2 & -1 & -3 & -2 & -3 & -2 & -1 & -2 &  0 &  0 &  \wgre{6}\\
\wor{Ile} & -2 & -1 &  0 & -2 & -1 & -3 & -2 & -2 & -2 & -2 & -2 & -2 & -2 &  \wgre{2} &  \wgre{5}\\
\wor{Leu} & -6 & -3 & -2 & -3 & -2 & -4 & -3 & -4 & -3 & -2 & -2 & -3 & -3 &  \wgre{4} &  \wgre{2} &  \wgre{6}\\
\wor{Val} & -2 & -1 &  0 & -1 &  0 & -1 & -2 & -2 & -2 & -2 & -2 & -2 & -2 &  \wgre{2} &  \wgre{4} &  \wgre{2} &  \multicolumn{1}{r}{\wgre{4}}\\
\cline{1-18}
\wor{Phe} & -4 & -3 & -3 & -5 & -4 & -5 & -4 & -6 & -5 & -5 & -2 & -4 & -5 &  0 &  1 &  2 & -1 &  \wgre{9}\\
\wor{Tyr} &  0 & -3 & -3 & -5 & -3 & -5 & -2 & -4 & -4 & -4 &  0 & -4 & -4 & -2 & -1 & -1 & -2 &  \wgre{7} & \wgre{10}\\
\wor{Trp} & -8 & -2 & -5 & -6 & -6 & -7 & -4 & -7 & -7 & -5 & -3 &  2 & -3 & -4 & -5 & -2 & -6 &  \wgre{0} &  \wgre{0} & \multicolumn{1}{r}{\wgre{17}}\\
\hline
\end{tabular}
\end{minipage}%
\begin{minipage}[t]{.7\textwidth}\vspace{0pt}
\raggedright\tiny
\begin{tabbing}
\hspace*{.2\textwidth}\=\hspace*{.15\textwidth}\=\hspace*{.3\textwidth}\=\hspace*{.6\textwidth}\= \kill
 \parb{.9\textwidth}{macierz punktacji podstawień
 aminokwasowych $\mathcal{S}$ jest symetryczna}\\

 \> \parb{.9\textwidth}{średnia ocena podstawień w obrębie
 \wgre{wydzielonych grup} reszt aminokwasowych jest wyraźnie większa
 od zera, co wskazuje, że tego typu wymiany są częste w sekwencjach
 homologicznych}\\

 \> \> \parb{.7\textwidth}{w obrębie grupy, reszty aminokwasowe
 wykazują podobieństwo własności fizykochemicznych}\\[1ex]

 \> \> \> \normalsize$\mathcal{S}_{ij} = 10 \log_{10} \frac{p(i,j|H)}{p(i,j|R)}$\\[2ex]

 \> \> \> \tiny\parbox{15em}{\raggedright częstość podstawień w sekw. homologicznych w~porównaniu z sekw. niespokrewnionymi}\\[-1ex]

 \> \> \> \parbox{.65\textwidth}{\tiny\begin{center}\renewcommand{\arraystretch}{1.1}\begin{tabular}{ r | r l }
$\mathcal{S}_{ij}$ & \multicolumn{2}{c}{$\wbl{\frac{p(i,j|H)}{p(i,j|R)}} = 10^{\mathcal{S}_{ij}/10}$}\\
& & \\[-2ex]
\hline
-7 & \wbl{0.2} & mniejsza\\
+3 & \wbl{2.0} & większa\\
0  & \wbl{1.0} & taka sama
\end{tabular}\end{center}}
\end{tabbing}
\end{minipage}
\end{frame}

\begin{frame}
 \frametitle{PAM}
 \framesubtitle{\wgre{jako seria macierzy podstawień aminokwasowych
 (MPA)\newline albo jednostka czasu ewolucji}}
\scriptsize

 Poszczególne macierze PAM zostały zoptymalizowane w celu możliwie
 najbardziej biologicznie poprawnego porównywania sekwencji o
 zróżnicowanej odległości ewolucyjnej

\begin{minipage}[t]{.3\textwidth}\vspace{0pt}

\begin{center}
\includegraphics[width=\textwidth]{images/matrixprod}
\end{center}

\begin{tabular}{r | r}
odległ. & \\
ewol.   & ID [\%]\\
w PAM   & \\
\hline
30 & 75 \\
80 & 50 \\
120 & 38 \\
250 & 20
\end{tabular}
\end{minipage}\hspace{.05\textwidth}
\begin{minipage}[t]{.64\textwidth}\vspace{0pt}
\begin{itemize}
 \item kolejne mnożenia PAM1 przez siebie: \\[2ex]

 Zmiany w długim czasie są \wor{ekstrapolowane} na podstawie zmian w
 krótkim czasie co prowadzi jednak do propagacji i \wbl{kumulowania
 błędów} wynikających z ograniczeń przyjętego modelu podstawień
 aminokwasowych\\[2ex]

 Np. przy porównywaniu blisko spokrewnionych sekwencji obserwowane
 podstawienia aminokwasowe wynikają głównie (80\%) ze zmian
 pojedynczego nukleotydu, podczas gdy dla dłuższych czasów dominujące
 stają się zmiany dwu- i trzy nukleotydowe
\end{itemize}
\end{minipage}\vfill~

\end{frame}

\begin{frame}
\frametitle{Względna entropia MPA}

\begin{minipage}[t]{.6\textwidth}\vspace{0pt}

\begin{itemize}\scriptsize

 \item wielkość entropii względnej $H$ MPA oznacza średnią ocenę
 pojedynczej pozycji dopasowania (nie mylić ze średną ważoną
 wartością MPA, która powinna być liczbą ujemną)

\end{itemize}

% \[ H = \sum_{i,\,j} q_{ij}\:s_{ij} = \sum_{i,\,j} q_{ij} \log_2
% \frac{q_{ij}}{p_i\,p_j} \]

 \vspace{-2ex}\relsize{-1}\[ H = \sum_{i,\,j} p(i,j|H)\:s_{ij} = \sum_{i,\,j}
 p(i,j|H) \log_2 \frac{p(i,j|H)}{p(i,j|R)} \]

\begin{itemize}\scriptsize

 \item znajomość średniej ilości informacji na pojedynczej pozycji
 dopasowania \mbox{(\wor{$H$})}, umożliwia wyznaczenie najmniejszej
 długości MSP \mbox{(\wor{$\min |{\rm{MSP}}|$})} koniecznej do uznania
 za istotne statystycznie wyników wyszukiwania przeprowadzonego z
 wykorzystaniem określonej MPA

\end{itemize}

\end{minipage}\hspace{.01\textwidth}%
\begin{minipage}[t]{.38\textwidth}\vspace{0pt}

\tiny\renewcommand{\arraystretch}{1.1}\sffamily
\begin{tabular}{ r | c | r }

 \parbox{5em}{\centering PAM distance} & \wor{$H$} [bit] &
 \parbox{5em}{\centering \wor{$\min |{\rm{MSP}}|$} (30 bit)}\\[1.5ex]
\hline
0   & 4.17 & 8\\
20  & 2.95 & 11\\ 
40  & 2.26 & 14\\
60  & 1.79 & 17\\
80  & 1.44 & 21\\
100 & 1.18 & 26\\
120 & 0.98 & 31\\
140 & 0.82 & 37\\
160 & 0.70 & 43\\
180 & 0.60 & 51\\
200 & 0.51 & 59\\
220 & 0.45 & 68\\
240 & 0.39 & 78\\
260 & 0.34 & 89\\
280 & 0.30 & 100\\
300 & 0.27 & 113\\
320 & 0.24 & 127\\
340 & 0.21 & 141\\
\end{tabular}

\end{minipage}

\begin{itemize}\scriptsize

 \item dla dużych wartości $H$, już względnie krótkie dopasowania
 (MSP) mogą być uznane za istotne statystycznie

\end{itemize}
\end{frame}

\begin{frame}
\frametitle{PAM: wpływ wyboru MPA na wynik dopasowania}

\begin{minipage}[t]{.3\textwidth}\vspace{0pt}
\raggedright

%rysunek 4 z BioTechniques 2000: pelne pokrycie zakresu dystansów
%ewolucyjnych z wykorzystaniem 3 macierzy

\begin{center}
\color{black}
\includegraphics[width=\textwidth]{images/pam}
\end{center}

\vspace*{-2ex}\begin{center}
\renewcommand{\arraystretch}{1.2}\scriptsize
\begin{tabular}{r | c}

 \parbox{4em}{\centering macierz PAM} & \parbox{4em}{\centering zakres
 długości}\\[1ex]
\hline
40 & 9$\div$21\\
120 & 19$\div$50\\
240 & 47$\div$123\\
\end{tabular}
\end{center}

\end{minipage}\hspace{.02\textwidth}%
\begin{minipage}[t]{.65\textwidth}\vspace{0pt}
\renewcommand{\arraystretch}{1.3}\scriptsize
 
 \wgre{\bf Tabela}\newline Średni wynik (w bitach) na pozycję
 dopasowania wyznaczanego z podaną macierzą PAM dla sekwencji o
 \wor{zadanej odległości ewolucyjnej}

\begin{center}\tiny
 \begin{tabular}{c | r r r r r r r r}

 \multirow{2}{*}{\parbox{4em}{\centering PAM matrix used}} &
 \multicolumn{8}{c}{Actual PAM distance $D$ of segments}\\ 

 & \wor{40} & 80 & \wor{120} & 160 & 200 & \wor{240} & 280 & 320\\[1ex]

 \hline

 \wor{40} & \wor{2.26} & 1.31 & 0.62 & 0.10 & -0.30 & -0.61 & -0.86 & -1.06\\
 80 & \wpi{2.14} & 1.44 & \wpi{0.92} & 0.53 & 0.23 & -0.02 & -0.21 & -0.37\\
 \wor{120} & 1.93 & 1.39 & \wor{0.98} & 0.67 & 0.42 & 0.22 & 0.06 & -0.07\\
 160 & 1.71 & 1.28 & \wpi{0.95} & 0.70 & 0.50 & 0.33 & 0.20 & 0.09\\
 200 & 1.51 & 1.16 & 0.90 & 0.68 & 0.51 & \wpi{0.38} & 0.26 & 0.17\\
 \wor{240} & 1.32 & 1.05 & 0.82 & 0.65 & 0.51 & \wor{0.39} & 0.29 & 0.21\\
 280 & 1.17 & 0.94 & 0.75 & 0.60 & 0.48 & \wpi{0.38} & 0.30 & 0.23\\
 320 & 1.03 & 0.84 & 0.68 & 0.56 & 0.46 & \wpi{0.37} & 0.30 & 0.24

 \end{tabular}
\end{center}
\end{minipage}

 \begin{itemize}\scriptsize

 \item przeszukiwanie jest efektywne jeśli różnica między bieżącą
 oceną jego wyników a oceną uzyskaną z wykorzystaniem optymalnie
 dobranej macierzy jest mniejsza od 2~bitów (4-krotna różnica w
 istotności statystycznej, 32/34$\approx$0.94)

 \item maksymalna ocena jeśli $D = M$; dla każdej macierzy $M$, im
 mniejszy rzeczywisty dystans ewolucyjny $D$ tym większa ocena
 dopasowania

 \end{itemize}

\end{frame}

\subsection{Algorytmy heurystyczne}


\begin{frame}
\frametitle{Szukanie homologów: problem klasyfikacji}

\raggedright\tiny
\begin{multicols}{2}

 \wor{czułość} (ang. \emph{sensitivity}) -- parametr ($C$) określający
 zdolność odszukania \wgo{wszystkich} sekwencji homologicznych

\vspace{-2ex}

\[ C = \frac{\rm{TP}}{\rm{TP} + \rm{FN}} \]

 \begin{itemize}

 \item wynik przeszukiwania jest zwykle \wgre{mniejszy} od
 oczekiwanego, ponieważ nie zawiera odległych homologów o marginalnym
 podobieństwie sekwencji

 \item sekwencje homologiczne, które nie zostały odszukane określane
 są jako wynik fałszywie ujemny (FN, ang.\ \emph{false
 \wgre{negatives}}) 

 \end{itemize}
%% specyficznosc/specificity
%% TN: wynik prawdziwie ujemny
%% TP: wynik prawdziwie dodatni

 \wor{selektywność} (swoistość, ang.\ \emph{selectivity}) -- parametr
 ($S$) określający zdolność odszukania \wgo{wyłącznie} sekwencji
 homologicznych

\vspace{-2ex}

\[ S = \frac{\rm{TP}}{\rm{TP} + \rm{FP}} \]

 \begin{itemize}

 \item wynik przeszukiwania może zawierać \wgre{dodatkowe} sekwencje
 niesłusznie uznane za homologi, tzw. wynik fałszywie dodatni
 (FP, ang. \emph{false \wgre{positives}})

 \end{itemize}

%%  \item[\wor{współczynnik korelacji}] -- parametr $k$ ($-1 \le k \le
%%  +1$, $-1$) będący miarą efektywności przewidywania homologii na
%%  podstawie wyników przeszukiwania bazy sekwencji w oparciu o określoną
%%  metodę heurystyczną: $-1$ oznacza, że wyniki wyszukiwania są zawsze
%%  błędne, $+1$ oznacza, że wyniki przewidywania są zawsze prawidłowe

%%\vspace{-3ex}

%% \[ k = \frac{\rm{TP} \cdot \rm{TN} + \rm{FP} \cdot \rm{FN}}
%%        {\sqrt{\rm{PP} \cdot \rm{PN} + \rm{FP} \cdot \rm{FN}}} \]

\begin{center}
\color{black}
\includegraphics[width=.7\columnwidth]{images/heuristic_evaluation}
\end{center}

\raggedright
\begin{description}

 \item[TP]: poprawnie zidentyfikowane homologii
 \item[FN]: niewyszukane homologii
 \item[FP]: sekwencje niehomologiczne błędnie opisane jako homologii
 \item[TN]: rzeczywiste sekwencje niehomologiczne

\end{description}
\end{multicols}
\end{frame}




\begin{frame}[fragile]
\frametitle{Wyniki przeszukiwania bazy danych sekwencji}

\raggedright\scriptsize
\begin{itemize}
 \item lista sekwencji uszeregowanych według malejącego podobieństwa
 do kwerendy wyznaczonego w oparciu o przyjęty system punktacji

 \item dla każdej z wyszukanych sekwencji podawany jest jej
 identyfikator, fragment opisu, długość sekwencji, punktowa ocena
 podobieństwa z kwerendą, ta sama ocena wyrażona w jednostkach
 bezwzględnych (bitach), wartość parametru $E$ (ang.\ \emph{E-value})

 \item wartość $E < 0.01$ \wor{sugeruje} istnienie homologii między
 sekwencją kwerendy a wyszukaną sekwencją bazodanową
\end{itemize}

\begin{center}
\begin{BVerbatim}[fontsize=\tiny]
PRIO_BOVIN P10279 bos taurus (bovine). major prion ( 264) 1430 266.0 5.6e-71
PRIO_CHICK P27177 gallus gallus (chicken). major p ( 273)  438  88.5 1.7e-17
K1CI_HUMAN P35527 homo sapiens (human). keratin, t ( 622)  206  47.3 9.3e-05
RB56_HUMAN Q92804 homo sapiens (human). tata-bindi ( 592)  194  45.1  0.0004
....
ROA1_MACMU Q28521 macaca mulatta (rhesus macaque). ( 319)  157  38.2   0.026
LEG3_CANFA P38486 canis familiaris (dog). galectin ( 295)  156  38.0   0.027
K1CJ_MOUSE P02535 mus musculus (mouse). keratin, t ( 569)  159  38.8    0.03
K1CM_MOUSE P08730 mus musculus (mouse). keratin, t ( 437)  157  38.4   0.032
K1CJ_BOVIN P06394 bos taurus (bovine). keratin, ty ( 526)  158  38.6   0.032

GRP8_ARATH Q03251 arabidopsis thaliana (mouse-ear  ( 169)  151  36.9   0.035
EGG2_SCHJA P19469 schistosoma japonicum (blood flu ( 207)  152  37.1   0.035
....
EGG1_SCHMA P06649 schistosoma mansoni (blood fluke ( 173)  127  32.6    0.69
SANT_PLAFV P09593 plasmodium falciparum (isolate v ( 375)  131  33.6    0.72
\end{BVerbatim}
\end{center}\vfill~
\end{frame}


\begin{frame}
\frametitle{Strategia przeszukiwania baz danych sekwencji}

\vspace*{-1ex}\begin{itemize}\raggedright\scriptsize

 \item bardzo szybkie wyznaczenie \wgre{przybliżonego} dopasowania
 lokalnego z każdą z sekwencji w bazie danych

 \item jeśli ocena takiego przybliżonego dopasowania jest odpowiednio
 wysoka, konstruowane są kolejno coraz bardziej dokładne dopasowania
 (im dokładniejszy algorytm wyznaczania dopasowania tym większa jego
 złożoność obliczeniowa, dlatego algorytm programowania dynamicznego
 uruchamiany jest wyłącznie zwykle dla kilkudziesięciu sekwencji
 najbardziej podobnych do sekwencji kwerendy)

%% szybkie (bo powierzchowne) przeszukanie bazy danych w celu
%% wyeliminowania sekwencji niepodobnych (do ``przeglądania'' bazy
%% używane są krótkie kilkuresztowe fragmenty kwerendy)
%% \item dopasowanie lokalne kwerendy kolejno z najbardziej podobnymi
%% sekwencjami

\item statystyczna ocena istotności wyników przeszukiwania bazy danych:
\end{itemize}

\vspace*{-2ex}%
\noindent\begin{minipage}[t]{.32\textwidth}\vspace{0pt}
\color{black}
\includegraphics[width=\textwidth]{images/evd}
% E(x) = 100*exp(-(x-30)/10-exp(-(x-30)/10))
\end{minipage}\hfill%
\begin{minipage}[t]{.65\textwidth}\vspace{0pt}
%\centering
%\includegraphics[width=.6\textwidth]{dp-matrix}
\begin{itemize}\scriptsize
\item jaka jest średnia liczba sekwencji, których podobieństwo jest
przypadkowe (analogia do relacji między szumem a sygnałem)
\item jak bardzo wyznaczona ocena podobieństwa między kwerendą a
odszukaną sekwencją bazodanową różni się od średniej oceny porównań z
innymi sekwencjami
\end{itemize}

\begin{description}\raggedright\relsize{-3}
\item[\wscgo{FastA}] W.R.~Pearson \& D.J.~Lipman (1988) Proc. Natl. Acad. Sci.
USA 85:2444--2448
\item[\wscgo{BLAST}] S.F.~Altschul, T.L.~Madden, A.A.~Sch\"{a}ffer, J.~Zhang,
Z.~Zhang, W.~Miller, D.J.~Lipman (1997) Nucleic Acids Res.
25:3389--3402
\end{description}
\end{minipage}\vfill~

%% \begin{multicols}{2}
%% \begin{description}\raggedright\relsize{-3}
%% \item[\wscgo{FastA}] W.R.~Pearson \& D.J.~Lipman (1988) Proc. Natl. Acad. Sci.
%% USA 85:2444--2448\columnbreak
%% \item[\wscgo{BLAST}] S.F.~Altschul, T.L.~Madden, A.A.~Sch\"{a}ffer, J.~Zhang,
%% Z.~Zhang, W.~Miller, D.J.~Lipman (1997) Nucleic Acids Res.
%% 25:3389--3402
%% \end{description}
%% \end{multicols}\vspace{\stretch{3}}
\end{frame}


\begin{frame}
\frametitle{\small Wpływ składu aminokwasowego
na oceny istotności statystycznej}

\begin{minipage}[c]{.3\textwidth}
\includegraphics[width=\textwidth]{images/prio_raw}
\end{minipage}\hspace{.01\textwidth}%
\begin{minipage}[c]{.38\textwidth}
\scriptsize

\[ P(z\ge{\cal{Z}}) = 1 -
  \exp(-\rm{e}^{-1.282{\cal{Z}} - 0.5772}) \]
\[ E(z\ge{\cal{Z}}) = DP(z\ge{\cal{Z}}) \]

\hrule

\[ P(z\ge7) < 7.11 \cdot 10^{-5} \]
\[ E(101529) < 7.22 \]

rzeczywistych homologów: $43$
\end{minipage}\hspace{.01\textwidth}%
\begin{minipage}[c]{.3\textwidth}
\includegraphics[width=\textwidth]{images/prio_pseg}
\end{minipage}

\vspace*{2ex}\noindent\begin{minipage}[t]{.54\textwidth}\vspace{0pt}
\raggedright\tiny Założenia modelu statystycznego, przyjęte
przy wyznaczaniu wartości \wgo{$E$} (ang. \wbl{\emph{\wgo{E}-value}}
$\leftrightarrow$ \wbl{\emph{\wgo{E}rror per query}},
\emph{false-positive rate}):
\begin{itemize}
\item oceny podobieństwa sekwencji tworzą rozkład wartości ekstremalnej
\item losowo wygenerowane sekwencje mają takie same własności jak
rzeczywiste sekwencje niespokrewnione 
\end{itemize}
W przypadku białek o \wgo{nietypowym składzie aminokwasowym}, ostatnie
założenie nie jest spełnione.
\end{minipage}\hspace{.01\textwidth}%
\begin{minipage}[t]{.45\textwidth}\vspace{0pt}\tiny
{\bfseries\ttfamily\color{bluish}\relsize{2}PRIO\_ATEPA P51446}\\[-1ex]
% pseg prio_atepa.fasta -p -c 46 -m 2
\begin{flushleft}\ttfamily
MANLGYWMLVLFVATWSDLGLCKKRPKPGG\\
WNTGGSRYPGQGSPGGNRY{\color{gold}PPQGGGWGQPH\\
GGGWGQPHGGGWGQPHGGGWGQPHGGGWGQ
AGG}THNQWNKPSKPKTNMKHM{\color{gold}AGAAAAGAV\\
VGGLGGYMLGSAMS}RPLIHFGNDYEDRYYR\\
ENMYRYPNQVYYRPVDQYNNQNNFVHDCVN\\
ITIKQH{\color{gold}TVTTTTKGENLTET}DVKMMERVVE\\
QMCITQYERESQAYYQRGSSMVLFS{\color{gold}SPPVI\\
LLISFLIFLIVG}
\end{flushleft}
\end{minipage}\vspace{\stretch{3}}
\end{frame}

\begin{frame}
\frametitle{Standaryzacja ocen lokalnego podobieństwa}

\begin{itemize}\scriptsize

 \item standardową jednostką podobieństwa pary sekwencji jest
 \wbl{BIT} (1~\tb{bit} odpowiada ilości informacji koniecznej do
 rozróżnienia między dwiema możliwościami, np.\ dzięki odpowiedziom
 \wbl{TAK}/\wbl{NIE} na odpowiednio sformułowane pytania; dla
 dokonania wyboru spośród $N$ możliwości potrzeba $\log_2N$ \tb{bitów}
 informacji):

 {\relsize{2}\[ E(s \ge S_{\rm{pkt}}) =
  \mathcal{K}mn\,{\rm{e}}^{-\lambda S_{\rm{pkt}}} \]}


{\relsize{2}\[ S_{\rm{bit}} = \frac{\lambda S_{\rm{pkt}} - {\rm{ln}}\,\mathcal{K}}{{\rm{ln}}\,2} \]}

{\relsize{2}\[ E(s\ge S_{\rm{bit}}) = \frac{mn}{2^{S_{\rm{bit}}}} \]}

\item wzrost oceny o 1~\tb{bit} powoduje dwukrotny wzrost istotności
statystycznej (tj.~$E$ staje się dwukrotnie mniejsze)
\end{itemize}

\end{frame}

\begin{frame}
\frametitle{Maskowanie sekwencji}

\scriptsize
\begin{itemize}

 \item wiele sekwencji \wor{aminokwasowych} i \wgre{nukleotydowych}
 zawiera powtarzające się odcinki sekwencji (powtórzenia tandemowe w
 DNA, struktury skręconych helis: \emph{coiled-coils}, sekwencje rozproszone
 (ALU), etc.)  lub fragmenty o niskiej złożoności składu

 \item jeśli sekwencja kwerendy zawiera taki fragment, za statystycznie
 istotne wyniki przeszukiwania bazy sekwencyjnej mogą być uznane
 dopasowanie z niespokrewnionymi ewolucyjnie sekwencjami 

\end{itemize}

\begin{center}\tiny
\renewcommand{\arraystretch}{1.3}
\begin{tabular}{l | p{30em} | p{7em}}
 \wgo{problem} & \wgo{opis} &
 \wgo{rozwiązanie}\tabularnewline
\hline
powtórzenia & \raggedright tandemowe, np. \wbl{CACACA}: VNTR
(\emph{Variable Numer of \wpi{Tandem Repeats}}, STR (\emph{Short Tandem
Repeats}), \wpi{sekwencje rozproszone}: SINE (\emph{Short Interspersed
Repetitive Element}, np.\ ALU), LINE, etc. & \wgre{DUST},
\mbox{\wgre{RepeatMasker}}, \wor{XNU}\tabularnewline

niska złożoność & \raggedright fragmenty sekwencji złożone z jendego
lub niewielkiej liczby typów reszt (\emph{\wpi{Low Complexity Regions}}:
LCR, np.\
\emph{leucine-rich}/\emph{proline-rich} regions), odcinki poly-A w
DNA & \wgre{DUST}, \wor{SEG} \tabularnewline

wektory & \raggedright fragmenty wektorów klonujących są często na
końcach, rzadziej w środku sekwencji bazowych/kwerendy &
\wgre{CrossMatch}, \wgre{VecScreen}

\end{tabular}
\end{center}

\end{frame}

\begin{frame}
\frametitle{Macierz podstawień vs.\ profile podstawień}

\begin{itemize}\scriptsize
 \item przeszukiwanie bazy, w trakcie którego podobieństwo sekwencji
 białkowych oceniane jest na podstawie macierzy podstawień, nader
 często nie pozwala na popawne zidentyfikowanie odległych homologów
 wykazujących marginalne podobieństwo do sekwencji kwerendy

 \item znacznie bardziej czułe jest wyszukiwanie oparte o tzw. profile
 podstawień (PSSM, \emph{Position Specific Scoring Matrix}), w których
 ocena podobieństwa/różnicy pary reszt aminokwasowych w porównywanych
 sekwencjach zależy nie tylko od rodzaju podstawienia (tj. np. $F
 \leftrightarrow W$) ale również od pozycji w sekwencji.

 Stąd mutacje (podstawienia/przerwy) w obrębie konserwatywnych
 rejonach centrów aktywnych białek cechuje zdecydowanie większy koszt
 niż zmiana tego samego typu mająca miejsce w
 funkcjonalno-strukturalnie neutralnym fragmencie sekwencji.

\end{itemize}

\begin{center}
\includegraphics[width=.7\textwidth]{images/pssm.png}
\end{center}
\end{frame}

\begin{frame}
\frametitle{PSI-BLAST}

\begin{itemize}\scriptsize
 \item metoda \emph{Position Specific Iterated Blast}
 pozwala na iteracyjne przeszukiwanie bazy sekwencyjnej z
 wykorzystaniem dynamicznie tworzonego profilu podstawień:

\begin{itemize}\scriptsize
 \item w pierwszej iteracji, podobieństwo sekwencji oceniane jest w
 taki sam sposób jak w metodzie BLAST (np. BLOSUM62, $\gamma(k) = 10 +
 k$)

 \item w oparciu o wyszukane sekwencje, których ocena podobieństwa
 charakteryzuje się współczynnikiem istotności statystycznej mniejszym
 od zadanej wartości progowej (zwykle $E < 0.01$), tworzony jest
 profil sekwencyjny używany w kolejnej iteracji
\end{itemize}
\end{itemize}

\begin{center}
\includegraphics[width=.7\textwidth]{images/psi_blast.png}
\end{center}

\end{frame}


\section{Wprowadzenie do programowania w Pythonie}


\begin{frame}
\frametitle{Obiekty}

\begin{itemize}\scriptsize

 \item Python jest językiem programowania \wbl{obiektowego}, co w
 praktyce oznacza, że w Pythonie \wgre{\bf wszystko jest obiektem}

 \item \wor{obiekt} jest abstrakcyjnym pojęciem odwołującym się do
 elementarnego składnika programu; utworzenie \wor{obiektu} prowadzi
 do pojawienia się jego \wor{instancji}, czyli wydzielonego w pamięci
 komputera obszaru zajmowanego przez dane tekstowe lub binarne oraz
 kod wykonywalny funkcji (tzw. \wor{metod}), które na nich działają

 \item obiektem w języku Python są zarówno zmienne typu tekstowego
 (\wbl{string}), zmiennoprzecinkowego (\wbl{float}), całkowitego
 (\wbl{int}) jak i funkcje, klasy (przepis definiujący obiekt) i
 moduły (oddzielne pliki tekstowe zawierające dane tekstowe i kody
 źródłowe funkcji lub klas)


 \item nazwa obiektu
 (\wpi{\tb{[a-zA-Z\_][a-zA-Z0-9\_]$\star\backslash$.py}}) nie może być
 na liście słów zastrzeżonych (\wbl{and}, \wbl{print},
 \wbl{import},..)


 \item elementy składowe danego obiektu to jego \wor{atrybuty};
 standardowa funkcja \wpi{dir(~)} wyświetla listę atrybutów danego
 obiektu

 \item poszczególne atrybuty obiektu są dostępne dzięki tzw. notacji z
 kropką, tj. \wor{\tb{ob\_name.att\_name}}

\end{itemize}
\end{frame}


\begin{frame}[fragile]
\frametitle{Operatory}

\begin{description}\scriptsize

 \item[przypisania] tożsamość obiektu (tzw.\ instancje) definiuje
 jego \wor{nazwa} (identyfikator) za pomocą \wor{operator}a
 przypisania \wpi{\tb{=}} (ang. \emph{assignment})

\end{description}
\begin{Verbatim}[fontsize=\scriptsize,codes={\catcode`$=3\catcode`^=7},
                 frame=single,framesep=3mm,commandchars=\\\{\}]
\wor{>>>} tekst \wpi{=} \wbl{'Witaj'}                                  \wgre{\# przypisanie zwykłe}
\wor{>>>} bruenet, ile_znakow \wpi{=} \wbl{'Witaj J-23'}, \wbl{len(tekst)}   \wgre{\# jednoczesne}
\wor{>>>} j23 \wpi{=} bruener \wpi{=} \wbl{'Stawka większa niż życie'}       \wgre{\# wielokrotne}
\wor{>>>} \wpi{bruenet.swapcase}()
'wITAJ j-23'
\end{Verbatim}
% $
\begin{description}\scriptsize
 \item[zawierania] in (np.\ \verb+print('A' in 'Abracadabra')+)
 \item[tożsamości] is (np.\ \verb+A is A+, \verb+A is B+)

 \item[matematyczne] dodawanie (\verb|3+(1+2j).real|), odejmowanie
 (\verb+4-2+), \mbox{mnożenie (\texttt{2*'ha'})}, dzielenie
 (\verb+10.2/3.8+), \mbox{reszta z dzielenia (\texttt{10\%3})}, potęgowania
 (\verb+2**8+)

\item[logiczne] not, or (\wbl{lub}), and, \wbl{albo}
\end{description}
\end{frame}

\begin{frame}[fragile]
\frametitle{Typy sekwencyjne}

\begin{description}\small
 \item[\wor{lista}] (ang. \emph{list}) pozwala na przechowywanie
 uporządkowanej sekwencji obiektów dowolnego typu. Listę definiuje się
 poprzez przypisanie wybranemu identyfikatorowi sekwencji obiektów
 ujętych w nawiasy \wpi{kwadratowe} i~rozdzielonych przecinkami.
\end{description}

\begin{Verbatim}[fontsize=\scriptsize,codes={\catcode`$=3\catcode`^=7},
                 frame=single,framesep=3mm,commandchars=\\\{\}]
\wor{>>>} a = [1,'ala',14,'aga',\wgre{$\backslash$}
\wgre{...}      10.2,'abra']
\wor{>>>} a
[1, 'ala', 14, 'aga', 10.2, 'abra']
\wor{>>>} a.sort()
\wor{>>>} a
[1, 10.2, 14, 'abra', 'aga', 'ala']
\wor{>>>} a.append([2,3.14]); print(a)
[1, 10.2, 14, 'abra', 'aga', 'ala', [2, 3.14]]
\wor{>>>} a[3] = a[3].upper(); a[2:5]
[14, 'ABRA', 'aga']
\wor{>>>} b = a[-5:-1]; b.reverse(); b
[[2, 3.14], 'ala', 'aga', 'ABRA', 14]
\end{Verbatim}
% $
\end{frame}

\begin{frame}[fragile]
\frametitle{Typy sekwencyjne}

\begin{description}\small
 \item[\wor{krotka}] (ang. \emph{tuple}) niemodyfikowalna lista (stała
 kolejność obiektów, wartości zmiennych). Składnia: \wpi{\tb{(a, b,
 ...)}} 
\end{description}
\begin{Verbatim}[fontsize=\scriptsize,codes={\catcode`$=3\catcode`^=7},
                 frame=single,framesep=3mm,commandchars=\\\{\}]
\wor{>>>} xyz = (1, 0.5, 'trzy')
\wor{>>>} xyz[-2:]
(0.5, 'trzy')
\wor{>>>} xyz[-1]
'trzy'
\wor{>>>} xyz[-1] = 3
Traceback (most recent call last):
  File "<stdin>", line 1, in ?
\wgre{TypeError}: object doesn't support item assignment
\end{Verbatim}
\end{frame}

\begin{frame}[fragile]
\frametitle{Szeregi arytmetyczne i pętle}

\hspace*{.03\textwidth}%
\begin{minipage}[t]{.57\textwidth}\vspace{0pt}
\begin{itemize}\small

 \item[range] jest niemutowalną sekwencją liczb całkowitych, często
 wykorzystywaną w~pętlach, składnia: \mbox{\wpi{\bf range([start,] stop[,
 step])}}

 \item w odróżnieniu od list, obiekt \wpi{range} zajmuje tę samę
 niewielką ilość pamięci operacyjnej niezależnie od rozmiaru/zakresu

 \item \wpi{\bf range} można określić jako \wor{leniwą} sekwencję liczb

 \item leniwość (wartościowanie leniwe, ang. \wor{lazy evaluation},
 przeciwieństwo: wartościowanie zachłanne/gorliwe (ang. \wor{eager
 evaluation})) w programowaniu oznacza obliczanie wyrażeń tak późno
 jak to możliwe

\end{itemize}
\end{minipage}\hfill%
\begin{minipage}[t]{.39\textwidth}\vspace{0pt}
\begin{Verbatim}[fontsize=\tiny,frame=single,label=\wgre{\bf range}]
>>> range(10)
range(0,10)
>>> list(range(10))
[0, 1, 2, 3, 4, 5, 6, 7, 8, 9]
>>> tuple(range(1,10))
(1, 2, 3, 4, 5, 6, 7, 8, 9)
>>> list(range(3,10,2))
[3, 5, 7, 9]
>>> list(range(10,3,-2))
[10, 8, 6, 4]
>>> r = range(0,15,3)
>>> r
range(0, 15, 3)
>>> 9 in r
True
>>> 11 in r
False
>>> r[-1]
12
>>> r[1:4:2]  # (3, 9)
range(3, 12, 6)
\end{Verbatim}

\end{minipage}
\end{frame}

\begin{frame}[fragile]
\frametitle{Typy sekwencyjne}
\begin{description}\small
 \item[\wor{ciąg tekstowy}] (ang. \emph{string}) sekwencja znaków
 ujęta między znakami apostrofu lub cudzysłowa/-ów
%% \item FIXME: raw, unixcode
\end{description}
%% raw, unicode

\begin{Verbatim}[fontsize=\scriptsize,codes={\catcode`$=3\catcode`^=7},
                 frame=single,framesep=3mm,commandchars=\\\{\}]
\wor{>>>} cool = 'hej ho'
\wor{>>>} ((cool[:3]+", ")*4).center(35)
'     hej, hej, hej, hej,      '
\wor{>>>} opis = \wgre{"""}
\wpi{     hej ho, hej ho, }
\wpi{do pracy by sie szlo \wgre{"""}}
... ... >>> print(opis)

\wpi{     hej ho, hej ho,}
\wpi{do pracy by sie szlo}
\end{Verbatim}
% $
\end{frame}


\begin{frame}[fragile]
\frametitle{Typ odwzorowujący: leksykony}

\begin{minipage}[t]{.49\textwidth}\vspace{0pt}
\raggedright\scriptsize
 dane w leksykonie (ang. \emph{dictionary}) zorganizowane są w
 taki sposób, że każdemu ze zdefiniowanych słów kluczowych (ang.
 \emph{\wgre{keys}}) odpowiada tylko jedna wartość (ang.
 \emph{\wpi{values}})

\begin{Verbatim}[fontsize=\tiny,codes={\catcode`$=3\catcode`^=7},
                 frame=single,framesep=3mm,commandchars=\\\{\}]
\wor{>>>} kesz = \{'\wgre{alicja}': \wpi{2150}, 
...         '\wgre{wiktoria}': \wpi{4500}, 
...         '\wgre{terminator}': \wpi{2150}\}
\wor{>>>} kesz.\wbl{keys}()
['wiktoria', 'alicja', 'terminator']
\wor{>>>} kesz.\wbl{values}()
[4500, 2150, 2150]
\wor{>>>} kesz['alicja']
2150
\wor{>>>} kesz[2150]
Traceback (most recent call last):
  File "<stdin>", line 1, in ?
\wbl{KeyError}: 2150
\end{Verbatim}

 \raggedright
 \wor{leksykon} nie może zawierać dwóch takich samych słów
 kluczowych i nie przechowuje informacji o~kolejności swoich
 elementów


\end{minipage}\hfill%
\begin{minipage}[t]{.49\textwidth}\vspace{0pt}
\scriptsize\raggedright
 \wgre{słowa kluczowe} leksykonów mogą być
 typu \wbl{string}, \wbl{int}, \wbl{float}, \wbl{tuple} ale
 np. nie \wbl{list}; \wpi{wartości} w leksykonie mogą być dowolnego
 typu

\begin{Verbatim}[fontsize=\tiny,codes={\catcode`$=3\catcode`^=7},
                 frame=single,framesep=3mm,commandchars=\\\{\}]
\wor{>>>} kesz['nostromo'] = 3500
\wor{>>>} kesz['alicja'] = 2750
\wor{>>>} \wbl{del}(kesz['terminator'])
\wor{>>>} kesz.\wbl{items}()
[('wiktoria', 4500), ('alicja', 2750), \wgo{$\rhd$}
                     ('nostromo', 3500)]
\wor{>>>} kesz.\wbl{has_key}('Nostromo')
0
\wor{>>>} kesz[20031017] = kesz.keys()
\wor{>>>} kesz.\wbl{get}(20031017)
\wpi{['wiktoria', 'alicja', 'nostromo']}
\wor{>>>} kesz[(4500,2750)] = $\backslash$
...      \{'naczelna':'wiktoria',
...       'szary pismak':'alicja'\}
\wor{>>>} x = kesz[20031017]
\wor{>>>} kesz[\wpi{x}] = 'to sie nie uda'
Traceback (most recent call last):
  File "<stdin>", line 1, in ?
\wbl{TypeError}: list objects are \wbl{unhashable}
\end{Verbatim}
\end{minipage}
\end{frame}


%% \begin{frame}
%% \frametitle{print czyli o drukowaniu słów kilka}
%% \end{frame}

%% \begin{frame}
%% biblioteka standardowa pythona, moduły i pakiety
%% \end{frame}

\subsection{Typy danych}

\subsection{Konstrukcje syntaktyczne}

\begin{frame}
W celu kontrolowania przepływu informacji w programie oraz sterowania
przebiegem programu wykorzystywane są:
\begin{description}
\item[konstrukcje warunkowe] (if..then..else), oraz
\item[konstrukcje iteracyjne] (for, while)
 \begin{itemize}\scriptsize \item iteracja (łac.\ \emph{iteratio}) --
 czynność powtarzania (najczęściej wielokrotnego) tej samej instrukcji
 (albo wielu instrukcji) w pętli. Mianem iteracji określa się także
 wszystkie operacje wykonywane wewnątrz takiej pętli.
 \end{itemize}
\hfill\tiny\url{http://pl.wikipedia.org/wiki/Iteracja}
\end{description}

\begin{description}

 \item[\wgre{Składnia}] każda z linii kodu w bloku poleceń
 wykonywanych w ramach struktur decyzyjnych i cyklicznych jest
 poprzedzona jednakową liczbą znaków odstępu lub tabulacji, stąd
 odpowiednie fragmenty kodu widoczne są jako \wor{bloki} poleceń.
\end{description}

\end{frame}

\begin{frame}[fragile]
\frametitle{Konstrukcje warunkowe}

\begin{minipage}[t]{.45\textwidth}\vspace{0pt}
\begin{itemize}\small
 \item umożliwiają sterowanie wykonywaniem programu

 \item ich podstawowym elementem są \wor{wyrażenia warunkowe}

 \item warunek jest spełniony jeśli odpowiednie \wor{wyrażenie
 warunkowe} zwraca liczbę różną od zera lub niepusty obiekt
\end{itemize}
\relsize{-2}\raggedright 
\wgre{Składnia}:\newline
\wbl{\tb{if~...}}\newline
\wbl{\tb{if~...~else~...}}\newline
\wbl{\tb{if~...~elif~...}}\newline
\wbl{\tb{if~...~elif~...~else~...}}
\end{minipage}\hspace{.02\textwidth}%
\begin{minipage}[t]{.52\textwidth}\vspace{0pt}
\begin{Verbatim}[fontsize=\scriptsize,codes={\catcode`$=3\catcode`^=7},
                 frame=single,framesep=3mm,commandchars=\\\{\},gobble=0]
\wor{>>>} name = 'Mozart'
\wor{>>>} \wbl{if} \wgre{name == 'Mozart'}\wbl{:}
...     print('Wolfgang Amadeusz',name)
... \wpi{elif} \wgre{name == 'Bach'}\wpi{:}
...     print('Jan Sebastian',name)
... \wpi{else:}
...     print(name)
...
Wolfgang Amadeusz Mozart
\wor{>>>}
\end{Verbatim}
% $
\end{minipage}
\end{frame}

\begin{frame}[fragile]
\frametitle{Operatory porównań i logiczne}

\begin{minipage}[t]{.45\textwidth}\vspace{0pt}
\begin{itemize}\small

 \item operatory
 porównań:\newline \tb{==}, \tb{<}, \tb{>}, \tb{<=}, \tb{>=}, \tb{!=} \item
 koniunkcja, alternatywa, alternatywa wykluczająca, negacja\\[1ex]

\scriptsize \hspace*{-.2\textwidth}\begin{tabular}{l l}
 X \wbl{and} Y &\parbox[t]{.3\textwidth}{\raggedright \wpi{\tb{prawda}} jeśli zarówno X jak i~Y jest \wpi{\tb{prawdą}}, \mbox{\wor{X \& Y}}}\\
 X \wbl{or} Y &\parbox[t]{.4\textwidth}{\raggedright\wpi{\tb{prawda}} jeśli X lub Y jest \wpi{\tb{prawdą}}, \mbox{\wor{ X $|$ Y}}}\\
 \wbl{operator.xor(}X, Y\wbl{)} & \parbox[t]{.4\textwidth}{\raggedright\wpi{\tb{prawda}} jeśli X \underline{albo} albo Y jest \wpi{\tb{prawdą}}, \mbox{\wor{X $^\wedge$ Y}}}\\
 \wbl{not} X &\parbox[t]{.4\textwidth}{\raggedright\wpi{\tb{prawda}} jeśli X jest \wpi{\tb{fałszem}}}
 \end{tabular}
 \end{itemize}
\end{minipage}\hspace{.01\textwidth}%
\begin{minipage}[t]{.53\textwidth}\vspace{0pt}
\begin{Verbatim}[fontsize=\scriptsize,codes={\catcode`$=3\catcode`^=7},
                 frame=single,framesep=3mm,commandchars=\\\{\},gobble=0]
\wor{>>>} a, name = 1920, 'Muchomorek'
\wor{>>>} if \wbl{1900 < a < 1930} and $\backslash$
...       name == 'Muchomorek':
...    print('Witaj Muchomorku')
...
'Witaj Muchomorku'
\wor{>>>} a = []
\wor{>>>} if a:  print('Lista nie jest pusta')
...
\wor{>>>} a.append(\wgre{None})
\wor{>>>} if a[0]:  print('Element okreslony')
...
\wor{>>>} if a[0] \wbl{is} None:  
...   print('Element jest \wgre{nie okreslony}')
...
Element jest nie okreslony
\end{Verbatim}
% $

\raggedleft
\tiny zobacz: \wbl{http://docs.python.org/library/operator.html\#mapping-operators-to-functions}
\end{minipage}
\end{frame}

\begin{frame}[fragile]
\frametitle{Konstrukcje iteracyjne}

\begin{minipage}[t]{.45\textwidth}\vspace{0pt}
\small\raggedright
 Bloki poleceń w programie wykonywane wielokrotnie (np.~w~ramach
 cyklów obliczeniowych~$\rightarrow$~\wbl{iteracji}).\\[1ex]

\begin{itemize}
 \item  pętla \tb{\wpi{while}} (dopóki \wgo{warunek} jest
 spełniony wykonuj instrukcje w bloku):

 \item pętla \tb{\wpi{for}} (dla kolejnych elementów listy,
 ciągu tekstowego, tupletu, innych obiektów (tych z~atrybutem
 \wbl{\tb{\_\_getitem\_\_}}) wykonuj instrukcje w bloku)
\end{itemize}
\end{minipage}\hspace{.02\textwidth}%
\begin{minipage}[t]{.52\textwidth}\vspace{0pt}

 \begin{Verbatim}[fontsize=\scriptsize,codes={\catcode`$=3\catcode`^=7},
                 frame=single,framesep=3mm,commandchars=\\\{\},gobble=1]
 \wor{>>>} start, end, step = 10, 50, 5
 \wor{>>>} \wpi{while} \wgo{start < end}:
 ...    print(start,\wpi{end=''})
 ...    start = start + step
 ... \wgre{else}:
 ...    print('\wbl{$\backslash$n}Koniec petli')
 ...
 10 15 20 25 30 35 40 45 
 Koniec petli
 \wor{>>>} \wbl{range(10,50,5)}
 [10, 15, 20, 25, 30, 35, 40, 45]
 \wor{>>>} \wpi{for} i \wpi{in} range(10,50,5):
 ...     print(i,end='')
 ... \wgre{else}:  print('$\backslash$nKoniec petli')
 ...
 10 15 20 25 30 35 40 45 
 Koniec petli
 \end{Verbatim}
 % $
\end{minipage}
\end{frame}


\begin{frame}[fragile]
\frametitle{Sterowanie w konstrukcjach iteracyjnych}
\begin{minipage}[t]{.45\textwidth}\vspace{0pt}\small

 \begin{description}
 \item[\wor{break}] przerywa wykonywanie instrukcji w pętli
 \item[\wor{continue}] rozpoczyna wykonywanie kolejnego cyklu w pętli

 \item[\wor{pass}] nie wykonuje żadnych działań. Jej użycie zwykle
 wynika z konieczności zastosowania się do reguł składniowych.
 \end{description}
\end{minipage}\hspace{.02\textwidth}%
\begin{minipage}[t]{.52\textwidth}\vspace{0pt}

 \begin{Verbatim}[fontsize=\scriptsize,codes={\catcode`$=3\catcode`^=7},
                 frame=single,framesep=3mm,commandchars=\\\{\},gobble=1]
 \wor{>>>} for i in range(10,50,5):
 ...      if i\wpi{\%}10: \wbl{continue}
 ...      print(i,end='')
 ...
 10 20 30 40
 \wor{>>>} while 1:
 ...      if i<=10: \wbl{break}
 ...      print(i,end='')
 ...      i = i - 5
 ...
 45 40 35 30 25 20 15
 \wor{>>>} while 1:
 ...    \wbl{pass} \wgo{\#} \wgre{Ctrl-C aby przerwac}
 ...
 Traceback (most recent call last):
   File "<stdin>", line 1, in ?
 \wgre{KeyboardInterrupt}
 \end{Verbatim}
 % $

\end{minipage}
\end{frame}

\begin{frame}
\frametitle{Funkcje}

\begin{itemize}\small
 \item w konwencji programowania strukturalnego, funkcja jest grupą
 (blokiem) poleceń zdefiniowaną przez programistę

 \item warto definiować funkcje, ponieważ:

\begin{itemize}\small

 \item zawierają implemenację procedur wykorzystywanych wielokrotnie w
 programie

 \item przy projektowaniu struktury programu, ich użycie pozwala na
 podzielenie złożonego problemu na oddzielne (prostsze do realizacji)
 zadania

 \item kod programu napisanego z użyciem funkcji jest bardzej czytelny
 i łatwiejszy do testowania w celu sprawdzania poprawności działania,
 lokalizacji i usuwania ewentualnych błędów (ang. \emph{debugging}),
 poprawiania efektywności (ang. \emph{profiling})

\end{itemize}
\end{itemize}
\end{frame}

\begin{frame}[fragile]
\vspace*{-1ex}\begin{minipage}[t]{.35\textwidth}\vspace{0pt}
\scriptsize
\begin{itemize}
 \item funkcję reprezentuje w~programie jej \wgo{nazwa}
 (identyfikator)
 
 \item funkcja może pobierać jeden lub więcej \wgre{argumentów} w
 postaci listy ujętej w parę nawiasów okrągłych i rozdzielonych
 przecinkami

 \item lista argumentów funkcji może zawierać \wbl{argumenty} podane
 \wbl{z} ich \wbl{domyślną wartością}

 \item funkcja zwraca \wpi{wynik} dowolnego typu lub pusty obiekt
 (\wbl{None})

\end{itemize}
\end{minipage}\hspace{.01\textwidth}%
\begin{minipage}[t]{.64\textwidth}\vspace{0pt}

\begin{Verbatim}[fontsize=\scriptsize,codes={\catcode`$=3\catcode`^=7},
                 frame=single,framesep=1mm,commandchars=\\\{\},gobble=0]
\wor{>>>} \wpi{def} \wgo{transcribe}\wgre{(dna)}\wpi{:}
...     \wbl{"""Return DNA string as RNA string, by}
...     \wbl{replacing T (thymine) with U (uracil)"""}
...     \wpi{return} dna.replace('T', 'U')
...
\wor{>>>} transcribe('CGAATATACT')
'CGAAUAUACU'
\wor{>>>} def baseContent(dna,\wbl{base='C'}, \wbl{norm=100}):
...     b = dna.count(base)
...     percent = \wbl{float}(b)/len(dna)*norm
...     return percent
...
\wor{>>>} myDNA = 'CGAATATACT'
\wor{>>>} baseContent(myDNA)
20.0
\wor{>>>} baseContent(myDNA,'A')
40.0
\wor{>>>} baseContent(myDNA, norm=1)
0.2
\wor{>>>} print(transcribe.\wgo{__doc__})
\wbl{Return DNA string as RNA string, by}
   \wbl{replacing T (thymine) with U (uracil)}
\wor{>>>} print(\wgre{'int:'},1//5, \wgre{'\wpi{$\backslash$t}float:'},1/5)
\wgre{int:} 0 	\wgre{float:} 0.2
\end{Verbatim}
% $
\end{minipage}
\end{frame}

\begin{frame}
\frametitle{Literatura}

\begin{itemize}\small

 \item The Python Tutorial (\wbl{\url{https://docs.python.org/3/tutorial/}})

 \item The Python Standard Library
 (\wbl{\url{https://docs.python.org/3/library/}})

 \item The Python
 HOWTO \mbox{(\wbl{\url{https://docs.python.org/3/howto/index.html}}),}
 w szczególności: \wor{Regular
 Expression}, \wor{Sorting}, \wor{argparse}, \wor{Fetch Internet
 Resources with urllib} a~dla bardziej zaawansowanych \wgre{Functional
 Programming}

 \item Jupyter Notebook
  Tutorial \mbox{(\wbl{\url{https://www.dataquest.io/blog/jupyter-notebook-tutorial/}})}

\end{itemize}
\end{frame}


\end{document}
